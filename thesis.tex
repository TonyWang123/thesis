\documentclass[degree=doctor,bibtype=numeric,degreetype=academic]{tongjithesis}
% 选项:
%   degree=[master|doctor], 							% 必选
%   bibtype=[numeric|authoryear], 						% 可选,数字式引用|作者-年份引用,默认为数字式(上标)引用
%   degreetype=[academic|profession|equaleducation],  	% 可选, 学术型|专业型|同等学力,默认为学术型
% 	electronic,                                 		% 可选, 电子版,(打印时删除)
%   secret,                                     		% 可选,是否保密,基本不用
%   pifootnote,                                 		% 可选,默认已打开
%   romantitle                                  		% 可选,默认已打开
%   注:默认已打开的选项可以使用arialtitle=false的形式关闭。

% 所有其它可能用到的包都统一放到这里了,可以根据自己的实际添加或者删除。
\usepackage{tongjiutils}
\usepackage{import}
\usepackage{algorithm}
\usepackage{algorithmic}
% \usepackage{graphicx}
% \usepackage{subfigure}
% \usepackage{amsmath} %equation
% \usepackage{mathabx} %rightrightharpoons




\newcommand{\para}[1]{\smallskip\noindent {\bf #1}}
\newcommand{\exampledp}{\texttt{ExampleDP}}
\newcommand{\rightrightharpoons}{{\Longrightarrow}}   % TODO: fix it
\newcommand{\codeword}[1]{\texttt{\small{#1}}}

\newcommand{\rulesep}{\unskip\ \vrule\ }
\newcommand \bnfdef  {\mathrel{\color{blue}{::=}}}
\newcommand \bnfalt  {\mathrel{\color{blue}{|}}}

\newcommand \keyfont {\mathsf}
\newcommand \ONPKT {\keyfont{onPacket(pkt)}}
% \newcommand \WHILE  {\keyfont{while}}
\newcommand \IFSTR     {\keyfont{if}}
\newcommand \THEN     {\keyfont{then}}
% \newcommand \SKIP   {\keyfont{skip}}
\newcommand \RETURNSTR   {\keyfont{return}}
% \newcommand \ADD   {\keyfont{add}}
% \newcommand \CONTAINS   {\keyfont{contains}}
% \newcommand \GET   {\keyfont{get}}
% \newcommand \PUT   {\keyfont{put}}
% \newcommand \GLOBAL   {\keyfont{global}}
% \newcommand \forl   {\keyfont{for}}
% \newcommand \while[2]   {\WHILE\ {#1}\ {#2}}
\newcommand \ifelse[3]  {\IFSTR\ {#1}:{#2}\ \THEN:{#3}}
\newcommand \return[1]   { \RETURNSTR\ {#1}}
% \newcommand \CYCLIC  {\keyfont{loop}}
% \newcommand \cyclic[1]   { \CYCLIC\ {#1}}

% \newcommand \JUMP  {\keyfont{jump}}
% \newcommand \CJUMP  {\keyfont{cjump}}
% \newcommand \LBL  {\keyfont{label}}

% \newcommand \jump[1]   { \JUMP\ {#1}}
% \newcommand \cjump[3]   { \CJUMP\ {#1}\ {#2}\ {#3}}
% \newcommand \lbl[1]   { \LBL\ {#1}}

\newcommand \tx {\mathtt{x}}
\newcommand \ty {\mathtt{y}}
\newcommand \tz {\mathtt{z}}
\newcommand \tg {\mathtt{g}}



\newtheorem{corrollary}{推论}

%参考文献更新使用biblatex包, 使用gb7714-2015标准, 具体参数设置可在cls文件中搜索biblatex进行了解
%加入bib文件(老版本文件依然能够使用)
\addbibresource{intro-cn/all-all.bib}   %


\begin{document}

% 定义所有的eps文件在 figures 子目录下
% \graphicspath{{figures/}}


%%% 封面部分
\frontmatter
\tongjisetup{
  %******************************
  % 注意:
  %   1. 配置里面不要出现空行
  %   2. 不需要的配置信息可以删除
  %******************************
  %
  %=====
  % 秘级
  %=====
  secretlevel={保密},
  secretyear={2},
  %
  %=========
  % 中文信息
  %=========
  % 题目过长可以换行(推荐手动加入换行符,这样就可以控制换行的地方啦)。
  ctitle={软件定义网络高级编程模型研究},
  cheadingtitle={软件定义网络高级编程模型研究},    %用于页眉的标题,不要换行 
  studentnumber={XXXXXXX},
  cdepartment={电子与信息工程学院},
  cmajorfirst={计算机科学与技术},
  cmajorsecond={计算机软件与理论},
  cauthor={XXX}, 
  csupervisor={XX 教授}, 
  % 如果没有副指导老师或者校外指导老师,把{}中内容留空即可,或者直接注释掉。
  % cassosupervisor={裴刚 教授~(校外)}, % 副指导老师
  % 日期自动使用当前时间,若需手动指定,按如下方式修改:
  % cdate={\zhdigits{2018}年\zhnumber{11}月},
  % 没有基金的话就注释掉吧。
  % cfunds={(本论文由我要努力想办法撑到两行的著名国家杰出青年基金 (No.123456789) 支持)},
  %
  %=========
  % 英文信息
  %=========
  etitle={Software Defined Network\\ High-Level Programming Model}, 
  edepartment={College of Electronics and Information Engineering},
  emajorfirst={Computer Science and Technology},
  emajorsecond={Computer Software and Theory},
  % 日期自动使用当前时间,若需手动指定,按如下方式修改:
  % edate={November,\ 2018},
  % efunds={(Supported by the Natural Science Foundation of China for\\ Distinguished Young Scholars, Grant No.123456789)},    
  eauthor={XXX},
  esupervisor={Prof. XXX},
  % eassosupervisor={Prof. Gang Pei (XiaoWai)}
  }

% 定义中英文摘要和关键字
\begin{cabstract}  

软件定义网络(SDN)的一个重要能力是高级编程。通过SDN高级编程,网络管理者可以实现对网络中数据流的灵活控制,而不必关心底层网络设备的具体架构。然而,一方面,随着SDN的普及,新应用的不断产生,人们对SDN高级编程的要求也不断提高;另一方面,随着网络设备技术的迅速发展,底层网络设备在不断变得高效、灵活的同时,其架构也变得越来越复杂。因此,如何将高级SDN程序高效地转化为底层网络设备的低级配置,是SDN高级编程模型研究始终需要考虑的问题。

对于单交换机数据通路的情况,目前有通过面向低级配置的接口生成具有固定结构(如OF-DPA)数据通路配置的相关工作,也有将高级程序转化为具有可定制结构(如RMT)数据通路配置的相关工作,但是没有将高级程序转化为具有固定结构数据通路配置的相关工作。要想利用那些高效的但具有固定结构数据通路,而且不想关心其复杂的实现架构,则将高级程序转化为具有固定结构数据通路配置的工作非常重要。而在进行转化前,一个重要问题则是,一个高级SDN程序是否可以被具有固定结构数据通路所实现。该问题的复杂地方在于高级SDN程序和底层数据通路之间结构差异性非常大,而且各自都具有丰富的多样性,因此我们需要一个系统的方法去解决该问题。针对该问题,我们提出一个将高级SDN程序和底层数据通路统一的特征空间,通过特征函数,高级SDN程序和数据通路都可以映射到该空间上的点,最后通过定义空间上点的比较函数,来判断该高级SDN程序是否可以被该数据通路实现。除此之外,当高级SDN程序具有循环结构时,如何将该程序高效地实现在可定制结构的数据通路上,也没有有效的工作。针对该问题,我们提出了重复软件流水线模型。该模型的优势在于可以对具有动态循环条件的循环结构进行转换,以及在转换后的结构上计算最佳数据通路时显示出较高的效率。

对于全网络数据平面的情况,大量的工作仍然是以响应式的模式(即不会主动生成配置,而等待数据包传递给SDN控制器)对网络进行管理,因此效率并不高。SNAP实现了从高级SDN程序经过主动编译转化为全网络且支持有状态的数据平面配置。然而,其采用One-Big-Switch模型,导致用户无法让数据包按指定的路径转发。因此,我们这里考虑的是一个非常灵活的高级SDN程序,即程序的输出为网络中的路径而非目的主机。并且,当网络存在具有固定功能的网络节点(如中间盒)时,我们的编程模型支持在高级SDN程序中调用相应网络节点功能对数据包进行处理,并根据调用返回的结果,用户可以选择网络中不同路径对数据包进行转发。对于这样的高级SDN程序,由于调用相应网络节点处理数据包时,数据包需要转发到该节点,而这可能会破坏程序中对数据包规定的转发路径的约束。因此,我们进一步考虑了其程序正确性问题,并提出系统路径约束来解决该问题。除此之外,针对不同网络场景,如软件定义联合网络的低时延要求,以及车联网中车的移动性要求,如何优化数据平面配置是一个挑战。为此,针对软件定义联合网络的场景,我们设计高级编程系统,并基于共享本地状态的方法,提高系统性能;针对车联网的场景,我们提出软件定义车联网的架构以及编程框架,并考虑车联网中车移动性的特点,提出优化的规则下发方法,减少生成的规则数量。

\end{cabstract}

\ckeywords{软件定义网络,可编程数据平面,高级编程模型}

\begin{eabstract}

A key capability of Software Defined Networking (SDN) is high-level programming. By using SDN high-level programming language, network managers can achieve flexible control of data flows in the network without the knowledge of the specific architecture of the underlying network devices. However, on the one hand, as SDN becoming more and more popular, new applications appears continuously. People's requirements for SDN high-level programming languages ​​are constantly improving; on the other hand, with the rapid development of network device technology, the underlying packet handling is becoming efficient and also flexible. However, at the same time, the data path architecture is becoming more and more complex. Therefore, how to efficiently convert high-level SDN programs into low-level configurations of underlying network devices is an issue that must be considered in SDN high-level programming model.



For the case of single switch data path, there are studies focusing on generating fixed-structure (such as OF-DPA) data path configuration through a low-level configuration interface, and also there are studies focusing on transforming a high-level program into a customizable structure (such as RMT) data path configuration. However, there is little work transforming a high-level program into a fixed-structured data path configuration To take advantage of efficient but fixed-structure data paths and do not want to care about their complex implementation architectures, it is important to turn high-level programs into the fixed-structured data path configurations. Then, an important issue before the transformation is whether a high-level SDN program can be implemented with a fixed-structure data path. This problem is not easy as the structural differences between the high-level SDN program and the underlying data path. So we need a systematic approach to solve the problem. To solve the problem, we propose a characteristic space that unifies the high-level SDN program and the underlying data path. Through the characteristic function, the high-level SDN program and the data path can be mapped to the points on the space. And by defining the comparison function of the points on the space, we can determine whether the high-level SDN program can be implemented by the data path. In addition, when the high-level SDN program has a loop structure, how to efficiently implement the program on the data path of the customizable structure is another problem. For this problem, we propose a repeated software pipeline model. The advantage of this model is that it can convert loop structures with dynamic loop conditions. And it is very efficient when calculating the optimal data path on the converted structure.




For the case of multiple data paths, a lot of studies are still in the reactive mode (that is, it does not actively generate the configuration, but waits for the packet to be delivered to the SDN controller) to manage the network, so the efficiency is not high. SNAP implements proactive compilation from a high-level SDN program to configurations of multiple stateful switches. However, it uses the One-Big-Switch model in which users cannot forward packets to a specified path in the network. What we are considering is a very flexible high-level SDN program where the user can calculate the path in the network as a return of the program. Moreover, when the network has a fixed-function node (such as a middlebox), our programming model supports invoking the corresponding node's function in the high-level SDN program to process packets, and based on the result returned by the invocation, the user can select different paths for packets to forward. For such a high-level SDN program, packets need to be forwarded to the node for the processing, which may break the constraints on the forwarding path specified in the program. Therefore, we further consider the correctness of the program and propose system path constraints to solve the problem. In addition, data plane configuration optimization is a challenge for different network scenarios, such as the low latency requirements of software defined coalition network and the mobility requirements of vehicles in the Internet of Vehicles. To this end, for the scenario of software defined coalition network, we design a high level programming system and improve the performance based on the method of sharing local state. For the scenario of Internet of Vehicles, we propose the Software Defined Internet of Vehicles architecture. For mobility in the network, the improved rule installation is proposed, by which the number of generated rules is reduced.

\end{eabstract}

\ekeywords{SDN, Programmable Datapath, High-Level Programming Model}
\makecover


% 目录
\tableofcontents
% 符号对照表
% \begin{denotation}
% \input{others/denotation}
% \end{denotation}

%%% 以下索引按需要选择
% 插图索引
% \listoffigures
% 表格索引
% \listoftables
% 公式索引
% \listofequations

%%% 正文 
\mainmatter
\import{./intro-cn/}{intro-chap.tex}
\import{./cap-cn/}{cap-chap.tex}
\import{./loop-cn/}{loop-chap.tex}
\import{./global-cn/}{global-chap.tex}
\import{./stateful-cn/}{stateful-chap.tex}
\import{./sdiv-cn/}{sdiv-chap.tex}
\import{./con-cn/}{con-chap.tex}



%%% 其它部分
\backmatter

% 致谢
% \begin{acknowledgement}
% 时光飞逝,岁月如梭。在同济大学攻读博士的这六年里所经历的、所收获的、所感悟的,都将是我一生最宝贵的财富。在论文即将完成之际,我衷心地感谢那些在博士期间给予我帮助、支持和鼓励的老师、同学、朋友和亲人。

在这里,首先我要感谢我的导师蒋昌俊教授。一直以来,蒋老师都以严格的要求和一丝不苟的作风对待学术研究,并对我的研究方向给予最大的支持和帮助。其次,我要感谢杨阳教授对我的精心指导。杨老师工作认真,为人正直,是我学习的榜样。他的言传身教将使我终生受益。再者,我要感谢王成教授对我的悉心关照。王老师对我的第一篇论文的逐句修改使我感受到了学术研究需要的认真态度。

感谢实验室博士后向乔对我研究的指导、鼓励和帮助,是他让我学会了对待任何事情都不能轻易放弃。感谢在美国期间和我一起进行研究的Christopher Leet,是他让我学会了认真态度是学术研究中不可缺少的因素。

感谢一起奋斗过的伙伴们。感谢李庚,高凯、张静轩、陈莘莘、钱亦辰、郭栋、杜海舟、林潇、余海涛、陈诗蔚、赵禹欣、刘炀、董舒、王君卓、顾忱、王浩然、胡佳园。

感谢我的父母,是他们对我的尊尊教导和无限支持,使我不断走在正确的人生道路上。

% \end{acknowledgement}
\clearpage
\newpage
\mbox{}
\newpage

% 参考文献
\printTJbibliography


% 附录
\begin{appendix}
% \input{data/appendix}
\end{appendix}

% 个人简历
\begin{resume}
\resumeitem{个人简历:}
\noindent 1989 年 11 月 3 日出生于吉林省长春市。\\
\noindent 2009 年 9 月考入同济大学计算机科学与技术系,2013年 7 月本科毕业并获得计算机科学与技术学士学位。\\
\noindent 2013 年 9 月免试进入同济大学计算机科学与技术系攻读计算机软件与理论博士学位至今。

\resumeitem{发表论文:} % 发表的和录用的合在一起
\begin{enumerate}[{[}1{]}]
\item \textbf{Wang, X.}, Wang, C., Zhang, J., Zhou, M., \& Jiang, C. Improved rule installation for real-time query service in software-defined internet of vehicles. IEEE Transactions on Intelligent Transportation Systems, 18(2), 225-235.
\item Leet, C., \textbf{Wang, X.}, Yang, Y. R., \& Aspnes, J. Toward the First SDN Programming Capacity Theorem on Realizing High-Level Programs on Low-Level Datapaths. In IEEE INFOCOM 2018-IEEE Conference on Computer Communications (pp. 711-719). IEEE. (共同一作)
\item \textbf{Wang, X.}, Kong, L., \& Jiang, C. Resolving the Loop in High-Level SDN Program for Multi-table Pipeline Compilation. In International Conference on Smart Computing and Communication (pp. 253-265). Springer, Cham.
\item \textbf{Wang, X.}, Xiang, Q., Tucker, J., Mishra, V., \& Yang, Y. R. Dandelion: A Novel, High-Level Programming System for Software Defined Coalitions with Local State Sharing. In MILCOM 2019-2019 IEEE Military Communications Conference (MILCOM). (已接收)
%\item Yang Y, Ren T L, Zhang L T, et al. Miniature microphone with silicon-
%  based ferroelectric thin films. Integrated Ferroelectrics, 2003,
%  52:229-235. (SCI 收录, 检索号:758FZ.)
%\item 杨轶, 张宁欣, 任天令, 等. 硅基铁电微声学器件中薄膜残余应力的研究. 中国机
%  械工程, 2005, 16(14):1289-1291. (EI 收录, 检索号:0534931 2907.)
%\item 杨轶, 张宁欣, 任天令, 等. 集成铁电器件中的关键工艺研究. 仪器仪表学报,
%  2003, 24(S4):192-193. (EI 源刊.)
%\item Yang Y, Ren T L, Zhu Y P, et al. PMUTs for handwriting recognition. In
%  press. (已被 Integrated Ferroelectrics 录用. SCI 源刊.)
%\item Wu X M, Yang Y, Cai J, et al. Measurements of ferroelectric MEMS
%  microphones. Integrated Ferroelectrics, 2005, 69:417-429. (SCI 收录, 检索号
%  :896KM.)
%\item 贾泽, 杨轶, 陈兢, 等. 用于压电和电容微麦克风的体硅腐蚀相关研究. 压电与声
%  光, 2006, 28(1):117-119. (EI 收录, 检索号:06129773469.)
%\item 伍晓明, 杨轶, 张宁欣, 等. 基于MEMS技术的集成铁电硅微麦克风. 中国集成电路, 
%  2003, 53:59-61.
\end{enumerate}

%\resumeitem{研究成果:} % 有就写,没有就删除
%\begin{enumerate}[{[}1{]}]
%\item 任天令, 杨轶, 朱一平, 等. 硅基铁电微声学传感器畴极化区域控制和电极连接的
%  方法: 中国, CN1602118A. (中国专利公开号.)
%\item Ren T L, Yang Y, Zhu Y P, et al. Piezoelectric micro acoustic sensor
%  based on ferroelectric materials: USA, No.11/215, 102. (美国发明专利申请号.)
%\end{enumerate}

\end{resume}

\end{document}
