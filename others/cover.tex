\tongjisetup{
  %******************************
  % 注意:
  %   1. 配置里面不要出现空行
  %   2. 不需要的配置信息可以删除
  %******************************
  %
  %=====
  % 秘级
  %=====
  secretlevel={保密},
  secretyear={2},
  %
  %=========
  % 中文信息
  %=========
  % 题目过长可以换行(推荐手动加入换行符,这样就可以控制换行的地方啦)。
  ctitle={软件定义网络高级编程模型研究},
  cheadingtitle={软件定义网络高级编程模型研究},    %用于页眉的标题,不要换行 
  studentnumber={XXXXXXX},
  cdepartment={电子与信息工程学院},
  cmajorfirst={计算机科学与技术},
  cmajorsecond={计算机软件与理论},
  cauthor={XXX}, 
  csupervisor={XX 教授}, 
  % 如果没有副指导老师或者校外指导老师,把{}中内容留空即可,或者直接注释掉。
  % cassosupervisor={裴刚 教授~(校外)}, % 副指导老师
  % 日期自动使用当前时间,若需手动指定,按如下方式修改:
  % cdate={\zhdigits{2018}年\zhnumber{11}月},
  % 没有基金的话就注释掉吧。
  % cfunds={(本论文由我要努力想办法撑到两行的著名国家杰出青年基金 (No.123456789) 支持)},
  %
  %=========
  % 英文信息
  %=========
  etitle={Software Defined Network\\ High-Level Programming Model}, 
  edepartment={College of Electronics and Information Engineering},
  emajorfirst={Computer Science and Technology},
  emajorsecond={Computer Software and Theory},
  % 日期自动使用当前时间,若需手动指定,按如下方式修改:
  % edate={November,\ 2018},
  % efunds={(Supported by the Natural Science Foundation of China for\\ Distinguished Young Scholars, Grant No.123456789)},    
  eauthor={XXX},
  esupervisor={Prof. XXX},
  % eassosupervisor={Prof. Gang Pei (XiaoWai)}
  }

% 定义中英文摘要和关键字
\begin{cabstract}  

软件定义网络(SDN)的一个重要能力是高级编程。通过SDN高级编程,网络管理者可以实现对网络中数据流的灵活控制,而不必关心底层网络设备的具体架构。然而,一方面,随着SDN的普及,新应用的不断产生,人们对SDN高级编程的要求也不断提高;另一方面,随着网络设备技术的迅速发展,底层网络设备在不断变得高效、灵活的同时,其架构也变得越来越复杂。因此,如何将高级SDN程序高效地转化为底层网络设备的低级配置,是SDN高级编程模型研究始终需要考虑的问题。

对于单交换机数据通路的情况,目前有通过面向低级配置的接口生成具有固定结构(如OF-DPA)数据通路配置的相关工作,也有将高级程序转化为具有可定制结构(如RMT)数据通路配置的相关工作,但是没有将高级程序转化为具有固定结构数据通路配置的相关工作。要想利用那些高效的但具有固定结构数据通路,而且不想关心其复杂的实现架构,则将高级程序转化为具有固定结构数据通路配置的工作非常重要。而在进行转化前,一个重要问题则是,一个高级SDN程序是否可以被具有固定结构数据通路所实现。该问题的复杂地方在于高级SDN程序和底层数据通路之间结构差异性非常大,而且各自都具有丰富的多样性,因此我们需要一个系统的方法去解决该问题。针对该问题,我们提出一个将高级SDN程序和底层数据通路统一的特征空间,通过特征函数,高级SDN程序和数据通路都可以映射到该空间上的点,最后通过定义空间上点的比较函数,来判断该高级SDN程序是否可以被该数据通路实现。除此之外,当高级SDN程序具有循环结构时,如何将该程序高效地实现在可定制结构的数据通路上,也没有有效的工作。针对该问题,我们提出了重复软件流水线模型。该模型的优势在于可以对具有动态循环条件的循环结构进行转换,以及在转换后的结构上计算最佳数据通路时显示出较高的效率。

对于全网络数据平面的情况,大量的工作仍然是以响应式的模式(即不会主动生成配置,而等待数据包传递给SDN控制器)对网络进行管理,因此效率并不高。SNAP实现了从高级SDN程序经过主动编译转化为全网络且支持有状态的数据平面配置。然而,其采用One-Big-Switch模型,导致用户无法让数据包按指定的路径转发。因此,我们这里考虑的是一个非常灵活的高级SDN程序,即程序的输出为网络中的路径而非目的主机。并且,当网络存在具有固定功能的网络节点(如中间盒)时,我们的编程模型支持在高级SDN程序中调用相应网络节点功能对数据包进行处理,并根据调用返回的结果,用户可以选择网络中不同路径对数据包进行转发。对于这样的高级SDN程序,由于调用相应网络节点处理数据包时,数据包需要转发到该节点,而这可能会破坏程序中对数据包规定的转发路径的约束。因此,我们进一步考虑了其程序正确性问题,并提出系统路径约束来解决该问题。除此之外,针对不同网络场景,如软件定义联合网络的低时延要求,以及车联网中车的移动性要求,如何优化数据平面配置是一个挑战。为此,针对软件定义联合网络的场景,我们设计高级编程系统,并基于共享本地状态的方法,提高系统性能;针对车联网的场景,我们提出软件定义车联网的架构以及编程框架,并考虑车联网中车移动性的特点,提出优化的规则下发方法,减少生成的规则数量。

\end{cabstract}

\ckeywords{软件定义网络,可编程数据平面,高级编程模型}

\begin{eabstract}

A key capability of Software Defined Networking (SDN) is high-level programming. By using SDN high-level programming language, network managers can achieve flexible control of data flows in the network without the knowledge of the specific architecture of the underlying network devices. However, on the one hand, as SDN becoming more and more popular, new applications appears continuously. People's requirements for SDN high-level programming languages ​​are constantly improving; on the other hand, with the rapid development of network device technology, the underlying packet handling is becoming efficient and also flexible. However, at the same time, the data path architecture is becoming more and more complex. Therefore, how to efficiently convert high-level SDN programs into low-level configurations of underlying network devices is an issue that must be considered in SDN high-level programming model.



For the case of single switch data path, there are studies focusing on generating fixed-structure (such as OF-DPA) data path configuration through a low-level configuration interface, and also there are studies focusing on transforming a high-level program into a customizable structure (such as RMT) data path configuration. However, there is little work transforming a high-level program into a fixed-structured data path configuration To take advantage of efficient but fixed-structure data paths and do not want to care about their complex implementation architectures, it is important to turn high-level programs into the fixed-structured data path configurations. Then, an important issue before the transformation is whether a high-level SDN program can be implemented with a fixed-structure data path. This problem is not easy as the structural differences between the high-level SDN program and the underlying data path. So we need a systematic approach to solve the problem. To solve the problem, we propose a characteristic space that unifies the high-level SDN program and the underlying data path. Through the characteristic function, the high-level SDN program and the data path can be mapped to the points on the space. And by defining the comparison function of the points on the space, we can determine whether the high-level SDN program can be implemented by the data path. In addition, when the high-level SDN program has a loop structure, how to efficiently implement the program on the data path of the customizable structure is another problem. For this problem, we propose a repeated software pipeline model. The advantage of this model is that it can convert loop structures with dynamic loop conditions. And it is very efficient when calculating the optimal data path on the converted structure.




For the case of multiple data paths, a lot of studies are still in the reactive mode (that is, it does not actively generate the configuration, but waits for the packet to be delivered to the SDN controller) to manage the network, so the efficiency is not high. SNAP implements proactive compilation from a high-level SDN program to configurations of multiple stateful switches. However, it uses the One-Big-Switch model in which users cannot forward packets to a specified path in the network. What we are considering is a very flexible high-level SDN program where the user can calculate the path in the network as a return of the program. Moreover, when the network has a fixed-function node (such as a middlebox), our programming model supports invoking the corresponding node's function in the high-level SDN program to process packets, and based on the result returned by the invocation, the user can select different paths for packets to forward. For such a high-level SDN program, packets need to be forwarded to the node for the processing, which may break the constraints on the forwarding path specified in the program. Therefore, we further consider the correctness of the program and propose system path constraints to solve the problem. In addition, data plane configuration optimization is a challenge for different network scenarios, such as the low latency requirements of software defined coalition network and the mobility requirements of vehicles in the Internet of Vehicles. To this end, for the scenario of software defined coalition network, we design a high level programming system and improve the performance based on the method of sharing local state. For the scenario of Internet of Vehicles, we propose the Software Defined Internet of Vehicles architecture. For mobility in the network, the improved rule installation is proposed, by which the number of generated rules is reduced.

\end{eabstract}

\ekeywords{SDN, Programmable Datapath, High-Level Programming Model}