\section{Pipeline capacity theorem}
\label{sec:pipeline-capacity-theorem}
We now present our pipeline capacity theorem. We develop our theorem in three sections.

We begin by parameterizing the amount of information a given function \textit{transmits} about its inputs to its outputs by introducing the notion of a function \textit{transmission vector set} ($tvs$) in Section~\ref{subsec:function-transmission}. We then proceed by parameterizing the amount of information a given pipeline \textit{can carry} about its inputs to its output by introducing the parallel notion of a pipeline \textit{capacity vector set} ($cvs$) in Section~\ref{subsec:pipeline-capacity}. Finally, we compare function $tvs$ with pipeline $cvs$ to develop a sufficient condition to test whether a function $f$ can be compiled into a pipeline $p$ in Section~\ref{subsec:capacity-theorem}. This test is our pipeline capacity theorem.

\subsection{Parameterizing function information transmission}
\label{subsec:function-transmission}

\para{Problem definition}

\para{Function information transmission:} We begin by offering the reader intuition into what the amount of information transmitted by a $f$ actually means. Consider the example function \texttt{onPkt} below:

\begin{verbatim}
def onPkt(Addr dstAddr, Port dstPort):
  if dstPort <= 1024:
    egress = stdHostTbl[dstPort]
  elif dstPort == 1025:
    egress = usrHostTbl[dstPort]
  else:
    return Drop();
  return Forward(egress);
\end{verbatim}

The domain of \texttt{onPkt}'s input \texttt{dstPort} is $2^{16}$, since the TCP protocol allows TCP sockets to take any port number between $0$ and $65535$. However, \texttt{onPkt} only depends very particularly on where \texttt{dstPort}'s value falls within this domain, specifically on whether \texttt{dstPort} is $\leq 1024$, $1025$, or $> 1025$. All other information about \texttt{dstPort} is irrelevant to \texttt{onPkt} - if we were to replace \texttt{dstPort} with the output of the encoder \texttt{encode(dstPort)}, which encodes \texttt{dstPort} with the code: $\{\leq 1024 \rightarrow 00, 1025 \rightarrow 01, >1025 \rightarrow 10\}$, \texttt{onPkt} could still be rewritten to run correctly. 

The existence of \texttt{encode(dstPort)} provides the following insight: if we had to transmit \texttt{onPkt}'s arguments to some black box function running \texttt{onPkt}, we could compress \texttt{dstPort} from 16 to 2 bits and still supply the black box with sufficient information to function.

\para{Equivalence classes:} We expand the proceeding idea into the notion of equivalence classes. Given an $f$ with inputs $M$, we say that two bindings $u$ and $v$ of some input $m_i \in M$ are equivalent if we could encode $m_i$ with an encoding that maps $u$ and $v$ to the same code-word, supply $f$ with this encoding instead of $m_i$, and still provide $f$ with enough information to run correctly. This is true if $f(m_i = u, M - m_i) == f(m_i = v, M - m_i)$ for any valid binding of $f$'s other inputs $M - m_i$. Extending this argument to sets of bindings $U$ and $V$ for sets of inputs $S \subseteq M$ gives us our definition of equivalence, below:

\vspace{1mm}
\noindent \textsc{Definition 1 - Equivalence:} Given an $f$ with inputs $M$, we say that two sets of bindings $U$ and $V$ are equivalent for some subset $S \subseteq M$ when $f(S = U, M - S) == f(S = V, M - S)$ for every valid binding of $M - S$.
\vspace{1mm}

Our definition of equivalency leads naturally to the definition of an equivalence class:

\vspace{1mm}
\noindent \textsc{Definition 2 - Equivalence class:} The set of all bindings of an $S \subseteq M$ which are mutually equivalent.
\vspace{1mm}

\para{Codewords:} Since an equivalence class's bindings are, by definition, equivalent, given a binding $b$ for an $S$ we can transmit a different binding from $b$'s equivalence class to an executor $e$ executing $f$ without changing $f$'s output. Such interchangability can be exploited to minimize $b$'s transmission bandwidth by assigning each of $S$'s equivalence classes a $codeword$, and transmitting $b$'s equivalence class's codeword in lieu of $b$, since $e$ can simply map $b$ back to some binding from $b$'s equivalence class to obtain enough information to execute $f$. We formalize our notion of codewords below:
\vspace{1mm}

\textsc{Definition 3 - Codewords:} A set of codes which correspond to each of an $S$'s equivalence classes.
\vspace{1mm}

Our notion of codewords allows us to find the minimum number of bits of information about an $S$ an executor requires to execute $f$, which we give as \textsc{Lemma 1}:
\vspace{1mm}

\noindent \textsc{Lemma 1:} If some $S$ has $k$ equivalence classes, an $e$ only requires $log_2(k-1)$ bits of information about $S$ to execute $f$ correctly.\vspace{1mm} 

\noindent \textit{Proof:} We can transmit any of $S$'s bindings $b$ to $e$ by transmitting $b$'s equivalence class's codeword. If $S$ has $k$ equivalence classes, we can assign its equivalence classes the codewords [$0$, $k-1$], which take at most $log_2(k-1)$ bits to transmit.
\vspace{1mm}

\para{Dataflow graphs:} Given \textsc{Lemma 1}, if we can determine an $S$'s equivalence classes, we can bound its transmission requirements. We can generate an $S$'s equivalence classes by transforming its $f$ into a dataflow graph.

\vspace{1mm}
\noindent \cleet{TODO: Describe DFG and f $\rightarrow$ dfg transformation.}

\para{Bounding equivalence class number:} Now that we have defined the notion of a dfg and shown how a dfg can be calculated from a $f$, we present our \textsc{Equivalence Class Bounding Theorem} (ECBT), which gives an upper bound on the number of equivalence classes an $S$ has.
\vspace{1mm}

\noindent \textsc{Equivalence Class Bounding Theorem:} An upper bound on $S$'s equivalence class number is given by the weight of the min-vertex-cut in $f$'s dfg $G$ separating $S$'s vertices $V_S$ from the set of vertices not solely descended from $S$, $D_{NS}$.
\vspace{1mm}

\noindent \textit{Proof:} We prove the ECBT using the following lemmas:
\vspace{1mm}

\noindent \textsc{Lemma 2:} Given a vertex cut disconnecting $D_{NS}$ from $V_S$ (that may cut $v \in V_S$), we can calculate $f$ without knowing $S$ if we know the bindings for the cut's vertices $C$.
\vspace{1mm}

\noindent \textit{Proof:}  We can calculate a dfg $G$'s output given bindings for G's roots because every vertex's variable in G must be descended from some subset of these roots. 

Consider the sub-graph of $G$, $G_s$ generated by removing every vertex separated from $D_{NS}$ by our vertex cut. Each of $G_s$'s roots is either a vertex corresponding to some $m_j \in M - S$ or $v \in C$. Therefore, given bindings for every vertex in $C$ and $m_j \in M - S$, we can calculate $G_s$'s output, and therefore $G$'s output. By the definition of a dfg $G$'s output is $f$'s output, and therefore we have shown that we can calculate $f$ given bindings for $C$ in lieu of bindings for $S$.

\vspace{1mm}
\noindent \textsc{Lemma 3:} If we can calculate $f$ by transmitting bindings for a set of variables $v \in V$ calculated using $S$ in lieu of bindings for $S$, $S$ has no more equivalence classes than $C$.
\vspace{1mm}

\noindent \textit{Proof:} Suppose, by way of contradiction, that a given $S$ had more equivalence classes than some $C$ calculated from it. Each of $S$'s equivalence classes' bindings must generate a binding from at least one of $C$'s equivalence classes. Therefore, by the pigeonhole principle, if $S$ has more equivalence classes than $C$, two bindings in different equivalence classes in $S$ must generate bindings from the same equivalence class in $C$. However, given that $f$'s output is agnostic to which binding in an equivalence class $C$ takes, the two bindings of $S$ must be in the same equivalence class which is a contradiction.
\vspace{1mm}

\noindent Given these lemmas, we prove ECBT as follows:
\vspace{1mm}

\noindent \textit{Proof:}
By \textsc{Lemma 2}, we can transmit a binding for the set of vertices in the cut $G.\textit{min-vertex-cut}(V_S, D_{NS})$ $C$ in lieu of a binding for $S$ to an executor and still calculate $f$. By, \textsc{Lemma 3}, $S$ can have no more vertices than $C$. We can put an upper bound on $C$'s number of equivalence classes by assuming that each of $C$'s bindings is a unique equivalence class. Since $C$ has $\sum_{v \in V}(|v.domain|)$ possible bindings, $S$ has at most $\sum_{v \in V}(|v.domain|)$ equivalence classes, which is value of $G.\textit{min-vertex-cut}(V_S, D_{NS})$.

The vertex cuts described by our ECBT can be used simultaneously, allowing an e to calculate $f$'s output without knowing $M$'s binding provided that it knows the bindings for the vertices from a set of vertex cuts that sever every $m_i \in M$ at least once. We formalize this property as \textsc{Lemma 4:}
\vspace{1mm}

\noindent \textsc{Lemma 4:} Given a set of vertex cuts $SC = (C_1, ..., C_n)$ each disconnecting $D_{NSi}$ from $V_{Si}$ for some $S$, if $\forall\ m_i \in M$ are included in at least one $S$, we can calculate $f$'s output if know the binding of every variable in $SC$'s cuts.
\vspace{1mm}

\noindent \textsc{Proof:} Consider the subgraph of $G$, $G_s$ generated by removing  every vertex separated from $D_{NSi}$ by $\forall\ C_i \in SC$. Each of $G_s$'s roots must either be a root of $G$ or a vertex from some cut, since any new roots formed in $G_s$ by removing each $C_i$'s ancestors must be vertices in $C_i$. Given that $\forall\ m_i \in M$  are separated from some $D_{NSi}$  by at least one $C_i$, no $m_i \in M$ can appear in $G_S$  that isn't part of a cut, and therefore  each of $G_s$'s roots must come from one of our cuts.

Given that we can calculate a dfg $G$'s output if we know that dfg's roots,  we can calculate $G_s$'s and therefore $G$'s and $f$'s outputs if we know bindings for every vertex in $SC$'s cuts.

\para{Transmission vectors:} To review, we can calculate an $f$'s transmission requirements for an $S$ by our ECBT and generate a test with them for whether an $e$ can execute $f$ by \textsc{Lemma 4}. We now formalize these transmission requirements using novel data structure that we term a $transmission vector$ (tv), which we define as follows:
\vspace{1mm}

\noindent \textsc{Transmission vector:} A transmission vector is a vector with a field $tv[S]\ \forall\ S \subseteq M$, such that if an $e$ receives $tv[S]$ bits about each $S$ in a set that collectively contains $\forall\ m_i \in M$, $e$ can compute $f$.
\vspace{1mm}

We calculate the value of each $tv[S]$ in a $tv$ as using our Function Transmission Theorem (FTT), which we give below:

\noindent \textsc{Function Transmission Theorem:} $tv[S] = \lceil log_2(k-1) \rceil$, where $k$ is the upper bound on a $f$'s equivalence classes given by the ECBT.
\vspace{1mm}

\noindent \textit{Proof:} By our ECBT, we have shown that $k$ is an upper bound  on an $S$'s equivalence classes. By \textsc{Lemma 1}, we have shown that we can transmit sufficient information about an $S$ with $k$ equivalence classes for an $e$ to execute $f$ using codewords with no more than $\lceil log_2(k-1) \rceil$ bits.

Finally, given that $S$'s $k$ equivalence classes correspond to  bindings of variables given by $S$'s min-vertex-cut, by \textsc{Lemma 4} if we transmit a set of such bindings for a set of $S$ which contain collectively contains $\forall\ m_i \in M$, $e$ can compute $f$.


%\noindent \textsc{Function Transmission Theorem:} An executor can execute $f$  if it receives $log_2(k_i)$ bits of information about every member of set of subsets of $M$: $SS = (S_1, ..., S_n)$, where each $S_i \in SS$ has $k_i$ equivalence classes and $\forall\ m_i \in M$ are contained in at least one $S_i$.
%%\vspace{2mm}




%\para{Codeword calculation:} sd
%%\vspace{2mm}

%\noindent \textsc{Lemma 3:} An $e$ can calculate a $S \subseteq M$'s binding's codeword if it receives codewords for a set of subsets of $M$ $PS = (P_1, ..., P_n)$, such that $\forall\ m_i \in S$ appear in at least one $Pi$.
%\vspace{2mm}

%\noindent \textit{Proof:} Suppose that $e$ has access to a set of maps from each $S \subseteq M$'s bindings to that $S$'s equivalence class's codewords (we discuss how such maps can be generated in Section III).

%$e$ can use these maps to map each $P_i$'s codeword to its equivalence class's  default binding and find values $\forall\ m_i \in S$ from at least one of these bindings, which can be mapped to $S$'s codeword. If an $m_i$ that appears in more than one $P_i$'s binding takes a different value in each binding, $e$ can chose arbitarily from these values because they come from equivalence classes of the same binding and are thus equivalent.
%\vspace{1.75mm}

%\noindent \textsc{Lemma 4:} If an $e$ has received sufficient information to calculate $M$'s codeword $e$, it can also simply execute $f$.
%\vspace{1.75mm}

%\noindent \textit{Proof:} If $e$ can calculate $M$'s codeword, $e$ can map it to a full set of $M$'s bindings which $e$ can use to execute $f$.

%\para{Transmission vectors:}

%\vspace{5mm}
%Our definition of an equivalence class, in turn, gives us a upper bound on the bits of information about an $S \subseteq M$ an executor requires to execute $f$, given as \textsc{Lemma 1}:

%\vspace{1.75mm}
%\noindent \textsc{Lemma 1:} If an $S \subseteq M$ has $k$ equivalence classes, we only need to transmit $log_2(k-1)$ bits of information about $S$ to an executor for it to execute $f$ correctly.
%\vspace{1.75mm}

%\noindent \textit{Proof:} We can encode any of $S$'s bindings by mapping each of $S$'s equivalence classes to an integer codeword $c$ between $0$ and $k-1$, and then transmitting the $c$ associated with its binding's equivalence class.

%Since $f$'s output is agnostic to which binding in an equivalence class $S$ takes, our executor can simply map $c$ back to any binding in its equivalence class and execute normally. If $c$ is at most $k-1$, transmitting it requires $log_2(k-1)$ bits.

%\para{Function transmission theorem:} We extend \textsc{Lemma 1} into our \textsc{Function Transmission Theorem}, which gives a sufficient condition to determine if an executor has received enough information about $M$ to execute $f$.
%%\vspace{1.75mm}

%\noindent \textsc{Function Transmission Theorem:} An executor can execute $f$  if it receives $log_2(k_i)$ bits of information about every member of set of subsets of $M$: $SS = (S_1, ..., S_n)$, where each $S_i \in SS$ has $k_i$ equivalence classes and $\forall\ m_i \in M$ are contained in at least one $S_i$.
%%\vspace{1.75mm}

%\noindent \textit{Proof:} If an executor receives an $SS$ in which $\forall\ m_i \in M$ appear at least once, this executor can recreate $M$'s binding since it can retrieve any $m_i \in M$'s value from at least one $S_i$.

%If we use \textsc{Lemma 1}'s encoding schema to transmit $SS$, transmitting any given $S_i$ only requires $log_2(k_i)$ bits. Under this schema, different $S_i$'s are permitted to provide different values for a given $m_i$, since an executor can map an $S_i$'s codeword $c$ to any binding in $c$'s  equivalence class, but by the definition of equivalence all of $m_i$'s values must come from the same equivalence class and thus the executor can select arbitrarily from any alternatives.

%\para{Transmission vectors:} We develop our \textsc{Function Transmission Theorem} into a novel data structure, the transmission vector (tv), which bounds $f$'s transmission requirements.
%%\vspace{1.75mm}

%\noindent \textsc{Definition 3 - Transmission vector:} A transmission vector (tv) is a vector with a field $tv[S] = log_2(k_i)\ \forall\ S \in M$ where $S_i$ has $k_i$ equivalence classes.
%\vspace{1.75mm}

%By the \textsc{Function Transmission Theorem}, an executor can execute an $f$ if it receives $tv[S_i]$ bits of data about each $S_i \subseteq M$ and $\forall\ m_i \in M$ are contained in at least one $S_i$.
%\vspace{1.75mm}

%\noindent \textit{Compressing transmission vectors:} By our definition of an $f$'s tv, a na\"{i}vely constructed tv contains $2^{|M|}$ fields0 since it contains a field for every $S \in M$. This na\"{i}ve approach, however, suffers from the disadvantage that practical routing $f$s may have large $M$ as modern SDN controllers increasingly route on multiple packet headers which may span multiple internet stack layers, blowing up their tv size.

%We avoid such exponential tv size increase by recognizing that a subset of $M$, $S = (m_i, ..., m_j)$'s inputs will rarely have fewer equivalence classes as a set than when separate, and thus we can omit and reconstruct any $tv[S]$ where $tv[S] = tv[m_i] + ... + tv[m_j]$.


%We develop \textsc{Lemma 1} into a novel data structure which we term a transmission vector $tv$, which defines an upper bound on the bits of data about a given $f$'s M an executor needs  to run $f$ correctly. 

%\noindent \textsc{Lemma 2:} An executor that receives $tv[m_i]$ bits of information about each $m_i \in M$ can execute $f$ correctly.
%\vspace{1.75mm}

%\noindent \textit{Proof:} By \textsc{Lemma 1} an executor only needs $log_2(k_i)$ bits of information about each $m_i$ to run correctly, and $tv[m_i] = log_2(k_i)$.
%\vspace{1.75mm}

%We can improve our $tv$'s upper bound on the bits of data required to execute an $f$ by adding fields for particular $S = (m_i, ..., m_j) \subseteq M$ such that $tv[S] = log_2(k)$ where $S$ has $k$ equivalence classes when $tv[S] < tv[m_i] + ... + tv[m_j]$. These fields can be leveraged if given opportunity to do precomputation before transmission.

%Such an $S$ occurs when $m_i, ..., m_j$ are correlated or when their reads are isolated from the rest of the reads in $f$. For example, in \texttt{onPkt}, \texttt{srcAddr} and \texttt{srcPort} are only read by the boolean function \texttt{isVerified}, and thus $tv[\texttt{srcAddr},\ \texttt{srcPort}] = 1$ bit (\texttt{isVerified}'s output), almost certainly less than $tv[\texttt{srcAddr}]$ + $tv[\texttt{dstAddr}]$.



%\noindent \textit{Proof:} Consider encoding $S$ by mapping each of its equivalence classes to some integer code word $c$, and transmitting that code word to our executor. $S$ only has $k$ equivalence classes, so transmitting $c$ only requires $log_2(k)$ bits. Since $f$ is agnostic to which binding in an equivalence class $S$ has, our executor can simply map $c$ back into any binding in $c$'s equivalence class and execute normally.

%\para{Function input encoding:} We use our notion of equivalence classes (ec) to develop an encoding schema that reduces the number of bits of information about $M$ an executor requires to execute $f$. This encoding schema, at a high level, has three major steps, which we present below:

%%\vspace{1.75mm}
%\begin{itemize}
%  \item First, we find the ec of a set of subsets of $M$ $(S_1, ..., S_n)$ ($SS$) where each $m_i$ in $M$ appears in at least one $S_i$, and then assign each $S_i$'s ec a unique integer as a codeword.
%
%  \item Second, given a binding for $M$, we find $S_i$'s ec under that binding, and transmit its codeword to our executor.
%
%  \item Finally, our executor reconstructs the transmitted binding by mapping each $S_i$'s codeword back to some binding for $S_i$ from that codeword's ec, and then picks values for each $m_i$ in $M$ arbitrarily from the reconstructed $S_i$.
%\end{itemize}
%%\vspace{1.75mm}

%\noindent \textit{Proof of correctness:} We now prove that our encoding schema provides the executor with enough information about $M$ to run $f$ correctly.
%%\vspace{1.75mm}

%\noindent \textit{Encoding transmission requirements:} Our encoding sch%ema's correctness allows us to bound the number of bits of information an executor requires about $M$ to execute $f$ as follows: 
%%\vspace{1.75mm}

%\noindent \textsc{Lemma 1:} If each $S_i$ in $SS$ has $k_i$ ec, we need to transmit $\sum_{i=1}^{n} log_2(k_i - 1)$ bits to transmit $SS$.
%%\vspace{1.75mm}

%\noindent \textit{Proof:} If each $S_i$ in $SS$ has $k_i$ ec, we can encode each $S_i$'s ec using an integer from $0$ to $k_i - 1$ as a codeword, which requires $log_2(k_i - 1)$ bits to transmit. Transmitting codewords $\forall\ S_i \in (S_1, ..., S_n)$ therefore requires $\sum_{i=1}^{n} log_2(k_i - 1)$ bits. 

%Equivalence classes have a useful property: they comprise a minimal set of  an $S \subseteq M$'s states an executor must distinguish to execute $f$. We formalize this property as \textsc{Lemma 1}:

%%\vspace{1.75mm}
%\noindent \textsc{Lemma 1:} An $S \subseteq M$'s equivalence classes defines a minimum set of states of $S$ an executor must distinguish between to execute $f$.
%%\vspace{1.75mm}

%\noindent \textsc{Proof:} Suppose that a given executor $e$ could distinguish fewer states of $S$ than $S$ has equivalence classes and still correctly execute $f$. Then, by the pigeonhole principle, $S$ must have two  equivalence classes that $e$ cannot distinguish, and therefore that $e$ must map to the same output. But, by the definition of an equivalence class, $f$ always outputs different results  for different equivalence classes, a contradiction.
%%\vspace{1.75mm}

%\para{Equivalence class representation:} Given a binding $b$ for an $S$, we can transmit a default binding  from $b$'s equivalence class, instead of $b$ to an executor $e$ executing $f$ without changing its output, since $f$ is agnostic to the binding in an equivalence class $S$ takes. More generally, to minimize transmission, we can assign each of $S$'s equivalence classes an integer, which we refer to as its $codeword$, and transmit $b$'s equivalence class's codeword to $e$ instead of $b$ without changing $e$'s output, since $e$ can map $b$'s codeword back to a default binding from $b$'s equivalence class and execute as before. We formalize our definition of codewords below:

%\textsc{Definition 3 - Codewords:} A $S$'s codewords are a set of integers that correspond to each of $S$'s equivalence classes.

%%\vspace{1.75mm}
%\noindent \textsc{Lemma 1:} If an $S$ has $k$ equivalence classes, an executor $e$  requires a minimium of $log_2(k-1)$ bits of information about $S$ to execute $f$ correctly. 
%%\vspace{1.75mm}

%LEM 1: we need only transmit $log_2(k)$ bits of information about $S$ to an $e$ executing $f$ to supply it with sufficient information to run correctly.

%\noindent \textit{Proof:}  By \textsc{Lemma 1}, an $S$'s equivalence classes comprise the minimum set of states $e$ must distinguish to execute $f$. The most compact way to transmit these states is to assign each a codeword from the set [$0$, $k-1$], which takes at most $log_2(k-1)$ bits to transmit.
%%\vspace{1.75mm}


\subsection{Parameterizing pipeline capacity}
\label{subsec:pipeline-capacity}


\para{Problem definition}

\para{Pipeline table capacity:} We begin examining pipeline capacity by considering the capacity of a single table $t_i$. We start by observing that we can view a $t_i$ as an executor that can execute any $f$ whose input domain is not too large by matching on each binding in $f$'s domain and outputting $f$'s value for that binding. This observation is formalized as \textsc{Lemma 2}:
\vspace{2mm}

\noindent \textsc{Lemma 2:} A pipeline table $t_i$ can execute any $f$ whose input domain size is less than the maximum number of rules $t_i$ can hold and only takes a subset of $t_i$'s match fields as inputs.
\vspace{2mm}

%\noindent \textit{Proof:} If an $f$'s domain size is smaller than the maximum number of rules a $t_i$ can hold, we can always execute  $f$ in that $t_i$ by populating $t_i$ with rules that match on each binding in $f$'s domain and output $f$'s value for that binding.

If $t_i$ is one of $p$'s output tables, it can can pass $f$'s output directly to $p$'s switch. If $t_i$ isn't, however, it must transmit information about $f$'s output for subsequent tables to act on. Specifically, $t_i$ can transmit information to subsequent tables $t_j \in p$ in two ways: by writing registers for $t_j$ to read, or by executing different pipeline branches with different $t_j$ depending on $f$'s output.

% , and such transmission may be limited by $p$'s architecture.

Each of these transmission mechanisms is limited: a $t_i$ with a $k$ bit output register can only transmit $2^k$ output values to its successors, whilst a $t_i$ that executes one of $k$ branches depending on $f$'s output can only transmit $k$ values. We, again, formalize this observation in \textsc{Lemma 3}:
\vspace{2mm}

\noindent \textsc{Lemma 3:} A $t_i \in p$ can transmit $log_2(t_i.$\branches$) + t_i.$\registerbits\ bits of information to subsequent $t_j \in p$,  where  $t_i.$\branches\ is the number of branches rooted at $t_i$, and $t_i.$\registerbits\ is $t_i$'s output registers' total bit length.
\vspace{2mm}

In practice, we find that of \textsc{Lemma 3}'s two transmission mechanisms, the capacity of branch encoding is dominated by the capacity of registers and therefore can be neglected. To support this assertion, we consider the standard pipeline architectures provided by two industry switches: OF-DPA and PicOS. Each architecture had less than 6 branches, which were only provided to handle different routing concerns (\eg\ single-cast vs multicast), whilst each contained 32 bit registers. Such limited opportunity for branch encoding in practical pipelines allows us to neglect it and still provide a tight lower bound on the information a table can transmit.

\para{Pipeline path dfgs:} Now that we have developed a mechanism to enumerate a given $p$'s paths, we describe how we convert each of these paths $r_p$ into a dfg $G$.

Before converting $r_p$ into a $G$, we transform $r_p$ into a path $r_s$ which computes exactly the same set of functions as $r_p$ by applying our \textsc{Table Split Transformation}.

\vspace{2mm}
\textsc{Table Split Transformation:} For each table $t_p \in r_p$, if $t_p$ has multiple outputs (registers or pipeline outputs) we split $t$ into a set of tables$T_s$ that write each of $t_p$'s outputs separately and share $t_p$'s inputs, which $r_s$ visits in arbitrary order.
\vspace{2mm}

To see that splitting a $t_p$ preserves the set of $f$ that it can execute, observe that  we can represent any given $t_p$'s contents in $t_s \in T_s$ by adding each of $t_p$'s rules whose outputs include a given $t_s$'s output to that $t_s$ and vice versa, and that we can traverse $T_s$ in arbitrary order because every $t_s$ shares $t_p$'s inputs, eliminating read-write conflicts.

Given our transformation of each $r_p \in p$, we now describe how we convert each transformed path $r_s$ into a pipeline dfg. We begin by describing our dfg, below:

\vspace{2mm}
\noindent \textsc{Pipeline dfg:} A pipeline dfg is a connected dag $G = (V, E)$ with exactly one leaf in which each $v \in V$ either represents a pipeline input, a register and its $t \in r_s$, or a pipeline output and its $t \in r_s$, distinguished by $v$'s field $v.type$ taking the values $M$, $R$, and $O$ respectively, where:
\begin{itemize}

 \item Pipeline input verticies ($v.type = M$) have a $v.name$ field which stores the pipeline input that $v$ represents. Iff a $v \in V$ is a pipeline input, it is a root of $G$.

 \item Register verticies ($v.type = R$) have fields $v.name$, $v.size$, and $v.S$, which store the register's name, its bit length, and the input verticies $v$ is descended from.

 \item Pipeline output verticies ($v.type = O$) have the field $v.S$, like register verticies, and are $G$'s only leaf.

\end{itemize}

\noindent Each edge $e$ in $G$ denotes that $e$'s target $v$'s table takes $e$'s source $v$'s pipeline input or register as an input.
\vspace{2mm}

\noindent \textit{Pipeline dfg generation:} To generate a pipeline dfg $G$ from a $r_s$, we simply step through $\forall\ t \in r_s$ in packet traversal order, adding $M$ verticies to $G$ for each unseen pipeline input $\in t$'s inputs, and a $R$ or $O$ vertex to $G$ for each $t$'s output, and edges between each $t$'s vertex and the verticies $t$ reads.

After generating $G$, we enumerate each $M$ vertex $v_m$'s descendents $d \in D$, and add $v_m.name$ to $d.S$.

\para{Capacity vectors:} We develop our observations about which $f$ a $t_i$ can execute and the information a $t_i$ can transmit into a test to determine if a $p$ can execute an $f$. We begin by considering $p$ whose table size is large, and then examine $p$ with limited size tables.

In the case that $\forall\ t_i \in p$ can have an arbitrary number of rules, by \textsc{Lemma 2} each $t_i$ can execute $f$ with arbitrary domain size. Execution in such a $p$ is still limited, however, by the information it can transmit about $m_i \in M_p$ to each $t_i$. 

\noindent \textsc{Capacity vector:} A capacity vector is a vector with fields $cv[ti.S]\ \forall\ t_i \in \rho$ whose value is the number of bits of information $t_i$ can output to $t_i.r$.
\vspace{2mm}

\noindent \textsc{The Pipeline Capacity Theorem:} Given a $p$ with inputs $M_p$ and cvs $cvs$ and a $f$ with inputs $M_f$ and tv $tv$, if both:
\begin{itemize}
  \item $M_f \subseteq M_p$ 
  \item $\exists\ cv \in cvs$ such that $cv[S] > tv[S]\ \forall\ S \in cv$
\end{itemize}
$\exists$ a set of contents of $p$ that can compute $f$.
\vspace{2mm}

\noindent \textit{Proof:} We prove the PCT with the following lemmas:

\noindent \textsc{Lemma x:} If $\forall\ t_i \in \rho$ from $t_1$ to $t_n$, $t_i.r.bits < tv[t_i.S]$, then each $t_i$ can output a codeword for $t_i.S$'s binding into $t_i.r$.
\vspace{2mm}

\noindent \textit{Proof:} We prove \textsc{Lemma x} by induction as follows:
\vspace{2mm}

\noindent \textit{Base case:} We will prove that if $t_1.r.bits < tv[t_1.S]$, then $t_1$ can output a codeword for $t_1.S$'s binding into $t_1.r$.

By \textsc{Lemma x}, $t_1$ can compute the codeword for the variables in $C_1$ and place it into $t_1.r$ if $t_1.inputs$'s dominate $C_1$. \cleet{Finish this}
\vspace{2mm}

\noindent \textit{Induction case:} By the induction hypothesis we assume that if $\forall\ t_i$ from $t_1$ to $t_k$, $t_i.r.bits < tv[t_i.S]$, then each $t_i$ can output a codeword for $t_i.S$'s binding into $t_i.r$. We show that if $t_{k+1}.r.bits < tv[t_{k+1}.S]$, $t_{k+1}$ can output a codeword for $t_{k+1}.S$ into $t_{k+1}.r$.

Computing $t_{k+1}.S$'s codeword is a three step process:
\begin{itemize}
  \item First, $t_{k+1}$ must compute values of the variables in $t_{k+1}.S$'s vertex cut $C_{k+1}$.

  \item Second, $t_{k+1}$ must map these values to their codeword.

  \item Third, $t_{k+1}$ must place this codeword in its register.
\end{itemize}

We show that $t_{k+1}$ can always accomplish steps 1 and 2 if $\forall\ t_i$ from $t_1$ to $t_k$, $t_i.r.bits < tv[t_i.S]$, and can accomplish step 3 if $t_{k+1}.r.bits < tv[t_{k+1}.S]$, proving the induction hypothesis.


By \textsc{Lemma x}, if an $e$ can calculate $C_{k+1}$ using $t_{k+1}.inputs$, $t_{k+1}$ can calculate $C_{k+1}$. Such an $e$ can calculate any vertex in a dfg given the values of that dfg's roots. Therefore, such an $e$ can calculate $C_{k+1}$  given the values of a set of verticies that dominate $C_{k+1}$ in $G$, since such verticies can form the roots of a subgraph of $G$ whose leaves are $C_{k+1}$.

We now prove that $t_{k+1}.inputs$ does dominate $C_{k+1}$. Given that $C_{k+1}$, by definition, is solely descended from $t_{k+1}.S$, to prove such domination it sufficies to show that every path from some $m_i \in t_{k+1}.S$ to $C_{k+1}$ includes some $v \in t_{k+1}.inputs$.

If $m_i \in t_{k+1}.S$, $t_{k+1}$ must either match directly on $m_i$ or $\exists$ some parent of $t_{k+1}$, $t_p$, such that $m_i \in t_p.S$.

If $t_{k+1}$ matches on $m_i$, then $m_i \in t_{k+1}.inputs$ and trivially every path from $m_i$ to $C_{k+1}$ includes some $v \in t_{k+1}.inputs$.


If $m_i \in$ some $t_p.S$, given that $t_p \in (t_1, ..., t_k)$, then by the induction hypothesis $t_p.r$ contains a codeword for the values of verticies which comprise a severing cut $C_p$ between $t_p.S$ and  all verticies in $G$ not solely descended from $t_p.S$, $D_{NSp}$.

If some $v \in C_{k+1}$ is not solely descended from $t_p.S$, $v \in D_{NSp}$ and by the definition of a severing cut ever path from $m_i$ to $v$ must include a vertex in $C_p$.

Otherwise, this $v \in C_{k+1}$ must lie on the minimium cut separating $t_p.S$ from $D_{NSp}$, which meands that $v \in C_p$ and again every path from $m_i$ to $v$ must include a vertex in $C_p$.

Given that $C_p \in t_i.inputs$, we have therefore shown that always $t_i.inputs$ dominates $C_{k+1}$. Therefore, an $e$, and by extension $t_{k+1}$, can always compute $C_{k+1}$.

Given that $t_{k+1}$ can always accomplish step 1, it can always accomplish step 2 by trivially outputing the codeword associated with $C_{k+1}$'s bindings rather than $C_{k+1}$ itself.

Trivially, this codeword can be placed in $t_i.r$, accomplishing step 3, if $t_{k+1}.r.bits < tv[t_{k+1}.S]$, and thus we have proved the induction hypothesis.
\vspace{2mm}

\noindent \textsc{Lemma Y:} If a $\rho$'s $t_n$ can output a codeword for $t_n.S$'s binding, $t_n$ can output $f$'s output.
\vspace{2mm}

\noindent \textit{Proof:} Given that a pipeline dfg is a connected graph with one leaf, $t_n.r$, $t_n.S$ must simply equal $M$. Any $e$ that knows $M$'s binding can execute $f$, and therefore by \textsc{Lemma x} if $t_n$ can compute a codeword for $M$'s binding, $t_n$ can compute $f$'s output.
\vspace{2mm}

\noindent Given these lemmas, we prove x as follows.
\vspace{2mm}

\noindent \textit{Proof:} By \textsc{Lemma x}, if $\forall\ t_i \in \rho$ from $t_1$ to $t_n$, $t_i.r.bits < tv[t_i.S]$, then each $t_i$ can output a codeword for $t_i.S$'s binding into $t_i.r$.

%In Section~\ref{subsec:function-transmission}, we showed that we could measure the amount of information that a given $f$ transmits about any subset of its inputs to its output using the notion of a $tvs$. In this section, we will measure the amount of information that a given $p$ transmits about any subset of its inputs to its output by introducing the notion of a $cvs$.%

%\para{Problem definition:} We define the problem of measuring the amount of information a given $p$ can carry about its inputs to its output as the pipeline capacity problem (PCP): given a pipeline $p$ with inputs $m_i$ in $M$, how much information can $p$ carry about a given subset of its inputs $S \in M$ to its outputs $e \in E$.%
%\vspace{2mm}
%\noindent \cleet{TODO: How do we go from here to capacity vector?}
%\vspace{2mm}

%\noindent \cleet{This assumes pipeline is a tree - need to add capacity vector set info.}

%\vspace{2mm}
%\noindent \textsc{Definition 4: Capacity Vector:} A capacity vector (cv) is a vector $cv[t_i.S] = t_i.$\registerbits\ $\forall\ t_i$ in a reduced register tree $r_r$ in a $p$.
%\vspace{2mm}

%Consider, for example, the $p$ \examplep\ below:

%\begin{verbatim}
%pExample: (NOTE: REDO THIS NICELY)
%t1:          t2:          t3:          t4:
%  m1 m2 | r1   m3 r1 | r2   m4 r1 | r3   r2 r3 | eg
%  ----------   ---------    ---------    ----------
%        |            |            |            |
%\end{verbatim}

%If \examplep's register $r_1$ is only 8 bits long, $t_1$, which writes $r_1$, can only transmit 8 bits of information about ($m_1$, $m_2$) to the rest of \examplep. If an $f$'s $tv[m_1, m_2] > 8$ bits, $t_1$ cannot transmit enough information about ($m_1$, $m_2$) to $t_2$ - $t_4$ to execute the remainder of $f$, and thus \examplep\ cannot execute $f$.

%\para{Understanding pipeline capacity:} As before, we begin by providing the reader with intuition into what the amount of information carried by a pipeline means.

%For $p$'s output to use information about a given pipeline input $m$, $p$ must transmit information about $m$ to one of its outputs. $p$ can transmit such information using one of three mechanisms.

%\textit{Table match fields:} $p$'s tables can transmit information about $m$ their output by either matching on $m$ or a register that contains information about $m$.

%\textit{Registers:} $p$ can transmit information about $m$ between tables if an initial table writes information about $m$ to a register for subsequent tables to read.

%\textit{Network encoding:} $p$ can implicitly transmit information about $m$ between tables by executing different branches depending on $m$'s value.

%Now that we have identified the mechanisms a given $p$ can use to transmit information, we parameterize the amount of information transmitted by each mechanism by introduce the notion of transmission capacity, defined below:

%\vspace{2mm}
%\textsc{Definition 3 - Transmission capacity:} A pipeline's transmission mechanism $t$'s capacity is equal to the number of equivalence classes of some subset of $t$'s inputs that $t$ can transmit.
%\vspace{2mm}

%We now use this definition to parameterize the capacity of each of the information transmission mechanisms defined above.

%\textit{Table match fields:} A table with $k$ rules can transmit $k$ equivalence classes by assigning each of its $k$ rules to a different equivalence class.

%\textit{Registers:} A $k$ bit register can transmit $2^k$ equivalence classes by using each of its $2^k$ values to represent a different equivalence class.

%\textit{Network encoding:} A $k$-way branch can transmit $k$ equivalence classes by executing a different branch for each equivalence class.

%In practice, we find that of the three mechanisms above, the capacity of network encoding is dominated by the capacity of match fields and registers. To support this assertion, we investigated the standard pipeline architectures provided by two industry switches, OF-DPA and PicOS. Each pipeline architecture had no more than 5 branches whilst allowing up to $2^32$ rules per table and $32$ bit registers. On average, therefore, a practical $p$'s network encoding capacity is 8 orders of magnitude smaller than the capacity of its registers or match fields, allowing us to safely neglect it.

%Given our measurement of match field and register transmission capacity, we now give a way to measure the transmission capacity of a $p$. We will begin by assuming that the size of $p$'s tables is large and thus that a pipeline's capacity is determined by the amount of information that its registers can carry between tables before extending our capacity measurement to limited size tables in Section III. 

%\noindent \textsc{Lemma 4:} An intermediate pipeline table $t_i$ can output $t_i.\#branches \times t_i.\#outputBits$ bits of information about its inputs $M_t$.
%\vspace{2mm}

%\noindent \textsc{Lemma 5:} A leaf pipeline table $t_{leaf}$ can output an arbitrary bits of information about its inputs $M_t$.
%\vspace{2mm}



%\subsection{Pipeline capacity theorem}
%\label{subsec:capacity-theorem}
%We now use our concepts of function $tv$ and pipeline $cvs$ to give our pipeline capacity theorem.

%\textsc{The pipeline capacity theorem:} Given a $f$ and $p$ with inputs $M_f$ and $M_p$ and tv/cvs $tv$/$cvs$ respectively, $\exists$ a set of contents with which we can populate $p$ that can compute $f$ if $M_f \subseteq M_p$ and $\exists\ cv \in cvs$ such that $cv[S] > tv[S]\ \forall\ S \in M_f$.

%\textit{Proof:} By the definition of a tv, a black box executor provided with $tv[S]$ bits of information about every $S \subseteq M_f$ can calculate $f$. The unlimited rule lookup table at the end of each path through $p$ is such an executor. By the definition of a cv, $\exists$ a set of contents $c$ which can provide this executor with $cv[S]$ bits of information about every $S \subseteq M_f$ where $cv[S] > tv[S]$, and thus a set of contents which can compute $f$.

%\para{Pipeline capacity theorem}

%\para{Execution path pipeline capacity theorem}