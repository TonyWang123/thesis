\section{Introduction}\label{sec:intro}
The recent success of software defined networking~(SDN) systems~\cite{swan, b4,
edgefabric} motivates the efforts to integrating SDN into military coalitions to
realize an efficient, agile, and optimal software-defined coalition (SDC)
infrastructure~\cite{vinod-sdc}. In SDC, autonomous coalition members operate
under highly dynamic tactical network environments with resource constraints,
such as limited power and processing capability, and dynamic connectivity. 

One may think that the integration of SDN into SDC is straightforward, because SDN
allows coalition members to realize efficient, flexible control over the
coalition networks through flexible packet match-action processing on the data
plane and logically centralized in the control plane~\cite{p4, rmt}.
Despite these promising features, they are insufficient for supporting SDC.

The fundamental reason behind this insufficiency is that SDC applications (\eg,
flexible load-balancing and detection of DNS amplification attack) have
stringent performance requirements (\eg, real-time, efficiency and reliability) 
under \textit{highly dynamic tactical network environments}. Implementing SDC
applications using the above SDN architecture in such environment would incur high communication overhead
between the data and control planes of SDC applications, resulting in
significant delay, efficiency and reliability issues.

Some SDC datapaths are recently proposed, which offload \textit{complex,
stateful} operations on packets (\eg, counting, flow security inspection and
congestion control) from the control plane to data plane
devices to improve the performance of SDC
applications~\cite{bianchi2014openstate, opensdc}.  Different datapath primitives, such as state
counter, packet buffer and packet in-network processing block, are designed in
these systems to support various SDC applications, \eg, stateful firewall,
proactive routing protection, and resilient routing. 

However, these systems
suffer from two limitations. First, results of offloaded stateful operations at
data plane devices, which we refer to as \textit{local state} in the remaining
of the paper, are not shared between data plane devices, resulting in
substantial resource under-utilization in SDC networks. For example, a firewall
middlebox is usually a bottleneck in an SDC network due to its
limited processing speed. As such, once a data
flow is identified as secure, non-sensitive by a firewall, all future packets of
this flow should be forwarded along a path with higher reliability and throughput, without the need
of passing the firewall again. However, the security state of this flow is only
stored in the firewall and not shared with other devices. 
Without knowing that this flow is non-malicious, an upstream
device (\eg, a gateway router) in the SDC network still has to forward all the
packets of this flow to the firewall. Second, although some distributed
update primitives provide interfaces for sharing of local state between data
plane devices~\cite{ddp}, configuring such low-level SDC datapaths still
requires time-consuming and error-prone manual efforts.
%Civilian high-level datapath programming systems, such as SNAP~\cite{snap}, do not support
%the sharing of local state between data plane devices. Though this may 
%work in more reliable civilian environments (\eg, campus and enterprise networks), 
%it can substantially hurt the performance of SDC in the highly-dynamic tactical environments.  

Toward addressing these two limitations, in this paper, we design \concept{}, a
novel, high-level SDC programming system. Specifically,
\concept{} introduces a series of novel programming primitives for users to
model and specify the behavior of SDC data plane devices. In addition, we design
a novel data structure called \textit{decision graph} and an efficient
configuration framework to translate high-level SDC programs into
efficient SDC datapath configurations. The translated configurations allow data
plane devices in SDC to exchange local states with the goal of efficiently
utilizing the resources in SDC networks (\eg, the packet processing capability
of devices and the data transmitting capability between devices). We implement a prototype of \concept{} and demonstrate its efficiency and efficacy using experiments. Results show that with \concept{}, the total
throughput can increase around two times higher than that of without
\concept{}.

The rest of this paper is organized as follows. Section~\ref{sec:motivation}
discusses related work and gives an motivating example. 
 Section~\ref{sec:system} gives an overview of \concept{} and Section~\ref{sec:details}
 presents the details of how \concept{} translates a high-level SDC program into data plane configurations. We evaluate the
 performance of \concept{} in Section~\ref{sec:eval} before concluding the paper
 in Section~\ref{sec:conclusion}.







