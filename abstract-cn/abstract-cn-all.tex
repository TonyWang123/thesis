\documentclass{ctexart}



\begin{document}

软件定义网络(SDN)的一个重要能力是高级编程。通过SDN高级编程语言,网络管理者可以实现对网络中数据流的灵活控制,而不必关心底层网络设备的具体架构。然而,一方面,随着SDN的普及,新应用的不断产生,人们对SDN高级编程语言的要求也不断提高;另一方面,随着网络设备技术的迅速发展,底层网络设备在不断变得高效、灵活的同时,其架构也变得越来越复杂。因此,如何将高级SDN程序高效地转化为底层网络设备的低级配置,是SDN高级编程模型始终需要考虑的问题。

首先,对于单交换机,目前有通过低级配置接口生成具有固定结构(如OF-DPA)数据通路配置的相关工作,也有将高级程序转化为具有可定制结构(如RMT)数据通路配置的相关工作,但是没有将高级程序转化为具有固定结构数据通路配置的相关工作。要想利用那些高效的但具有固定结构数据通路,而且不想关心其复杂的实现架构,则将高级程序转化为具有固定结构数据通路配置的工作非常重要。而在进行转化前,一个重要问题则是,一个高级SDN程序是否可以被具有固定结构数据通路所实现。该问题的复杂地方在于高级SDN程序和底层数据通路之间结构差异性非常大,而且各自都具有丰富的多样性,因此我们需要一个系统的方法去解决该问题。针对该问题,我们提出一个将高级SDN程序和底层数据通路统一的特征空间,通过特征函数,高级SDN程序和数据通路都可以映射到该空间上的点,最后通过定义空间上的比较函数,来判断该高级SDN程序是否可以该数据通路实现。除此之外,当高级SDN程序具有循环结构时,如何将该程序高效地实现在可定制结构的数据通路上,则并没有有效的工作。针对该问题,我们提出了重复软件流水线转换。,在计算循环结构的最佳数据通路结构时显示出更高的效率。


第一,在面向单交换机编程模型的研究中,已有的相关工作可以分为如下:1. 通过低级配置接口生成具有固定结构(如单流表或OF-DPA)数据通路的配置;2. 将高级程序转化为具有可定制结构(如RMT)数据通路的配置。但是缺少将高级程序转化为具有固定结构数据通路的配置的相关工作。而在转化之前,需要考虑的是该固定结构数据通路是否可以实现高级程序。因此,如何判断一个高级SDN程序是否可以在具有固定结构数据通路上实现,是一个挑战。除此之外,当SDN程序具有循环结构时,如何将该SDN程序实现在可定制结构数据通路上并没有相关工作。因此高级SDN程序中循环结构在可定制结构数据平面的高效实现,是一个挑战。

第二,在面向全网络编程模型的研究中,对于一般网络,大量工作是以响应式模式(即不会主动生成配置)对网络进行管理,因此效率并不高。SNAP实现了主动编译生成数据平面配置,但其程序并不灵活。因此,给定一个高级灵活的SDN程序,如何生成面向全网络的数据平面配置是一个挑战。除此之外,针对不同场景,如软件定义联合~\cite{mishra2017comparing}网络的低时延要求,以及车联网中车的移动性要求,如何优化数据平面配置是一个挑战。


\end{document}