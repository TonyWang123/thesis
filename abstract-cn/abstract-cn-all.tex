\documentclass{ctexart}



\begin{document}

软件定义网络(SDN)的一个重要能力是高级编程。通过SDN高级编程语言,网络管理者可以实现对网络中数据流的灵活控制,而不必关心底层网络设备的具体架构。然而,一方面,随着SDN的普及,新应用的不断产生,人们对SDN高级编程语言的要求也不断提高;另一方面,随着网络设备技术的迅速发展,底层网络设备在不断变得高效、灵活的同时,其架构也变得越来越复杂。因此,如何将高级SDN程序高效地转化为底层网络设备的低级配置,是SDN高级编程模型始终需要考虑的问题。

首先,对于单交换机数据通路的情况,目前有通过低级配置接口生成具有固定结构(如OF-DPA)数据通路配置的相关工作,也有将高级程序转化为具有可定制结构(如RMT)数据通路配置的相关工作,但是没有将高级程序转化为具有固定结构数据通路配置的相关工作。要想利用那些高效的但具有固定结构数据通路,而且不想关心其复杂的实现架构,则将高级程序转化为具有固定结构数据通路配置的工作非常重要。而在进行转化前,一个重要问题则是,一个高级SDN程序是否可以被具有固定结构数据通路所实现。该问题的复杂地方在于高级SDN程序和底层数据通路之间结构差异性非常大,而且各自都具有丰富的多样性,因此我们需要一个系统的方法去解决该问题。针对该问题,我们提出一个将高级SDN程序和底层数据通路统一的特征空间,通过特征函数,高级SDN程序和数据通路都可以映射到该空间上的点,最后通过定义空间上点的比较函数,来判断该高级SDN程序是否可以被该数据通路实现。除此之外,当高级SDN程序具有循环结构时,如何将该程序高效地实现在可定制结构的数据通路上,则并没有有效的工作。针对该问题,我们提出了重复软件流水线模型。该模型的优势在于可以对具有动态循环条件的循环结构进行转换。以及在转换后的结构上计算最佳数据通路时显示出更高的效率。

其次,对于全网络数据平面的情况,大量的工作仍然是以响应式的模式(即不会主动生成配置,而等待数据包传递给SDN控制器)对网络进行管理,因此效率并不高。SNAP实现了从高级SDN程序经过主动编译转化为全网络且支持有状态的数据平面配置。然而,其采用One-Big-Switch模型,即用户无法让数据包按指定的路径转发。因此,我们这里考虑的是一个非常灵活的高级SDN程序,其中用户可以算出网络中路径作为程序的返回。并且,当网络存在具有固定功能的网络节点(如中间盒)时,我们的编程模型支持在高级SDN程序中调用相应网络节点功能对数据包进行处理,并根据调用返回的结果,用户可以选择网络中不同路径对数据包进行转发。对于这样的高级SDN程序,由于调用相应网络节点处理数据包时,数据包需要转发到该节点,而这可能会破话程序中对数据包规定的转发路径的约束。因此,我们进一步考虑了其程序正确性问题,并提出系统路径约束来解决该问题。除此之外,针对不同网络场景,如软件定义联合网络的低时延要求,以及车联网中车的移动性要求,如何优化数据平面配置是一个挑战。为此,针对软件定义联合网络的场景,我们设计高级编程系统,并基于共享本地状态的方法,提高系统性能;针对车联网的场景,我们提出软件定义车联网的架构以及编程框架,并考虑车联网中车移动性的特点,提出优化的规则下发方法,减少生成的规则数量。

关键字:软件定义网络,可编程数据平面,高级编程模型


\end{document}